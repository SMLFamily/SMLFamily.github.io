\subsection{Types, Type Applications and Type functions} % (cvr)
\label{tyfun-sec}
A type $\tau$ is an {\sl equality type},\index{22.3} or {\sl admits equality}, if it is
of one of the forms
\begin{itemize}
\item $\alpha$, where $\alpha$ admits equality;
\item $\{\lab_1\mapsto\tau_1,\ \cdots,\ \lab_n\mapsto\tau_n\}$,
      where each $\tau_i$ admits equality;
% \item $\tauk\t$, where $t$ and all members of $\tauk$ admit equality; % (cvr)
\item $\apptype{\tauk}{\typeapp}$, where \typeapp\ and all members of $\tauk$ admit equality; % (cvr)

\item $(\tau')\REF$.\index{23.1}
\end{itemize}
% (cvr
(Note that  if $\tau$ is a package type \longpackagetype\ then 
it \emph{does not} admit equality.)

A type application $\typeapp$ is an {\sl equality type application},
\index{22.3} or {\sl admits equality}, if it is
of one of the forms
\begin{itemize}
\item $\t$, where $\t$ admits equality;
\item $\apptypeapp{\typeapp'}{\typefnc}$, where \typeapp\ and \typefnc\ both
       admit equality.
\end{itemize}

A type function $\typefnc$ is an {\sl equality type function},
\index{22.3} or {\sl admits equality}, if it is
of one of the forms
\begin{itemize}
\item \typetypefnc{\alphak}{\typefnc'}, where, when the type variables $\alphak$
      are chosen to admit equality, then $\typefnc'$ also admits equality;
\item \typefnctypefnc{\t}{\typefnc'}, where,  when the type name $\t$
      is chosen to admit equality, then $\typefnc'$ also admits equality;
\item \typeapptypefnc{\typeapp}, where \typeapp\ admits equality.
\end{itemize}

% cvr)
% (cvr
% \label{tyfcn-lab}
% A type function $\theta=\Lambda\alphak.\tau$\index{23.2}
%  has arity $k$; it must be
% {\sl closed} -- i.e.
% $\TyVarFcn(\tau)\subseteq\alphak$ -- and the bound variables must
% be distinct. Two type functions are considered equal
% if they only differ in their choice of bound variables (alpha-conversion).
% In particular, the equality attribute has no significance in a 
% bound variable of a type function; for example, $\Lambda\alpha.\alpha\to
% \alpha$ and $\Lambda\beta.\beta\to\beta$ are equal type functions
% even if $\alpha$ admits equality but $\beta$ does not.
% %poly 
% Similarly, the imperative attribute has no significance 
% in the bound variable of a type function.
% If $t$ has arity $k$, then we write $t$ to mean $\Lambda\alphak.\alphak\t$
% (eta-conversion); thus $\TyNames\subseteq\TypeFcn$. $\theta=\Lambda\alphak.\tau$
% is an {\sl equality} type function, or {\sl admits equality}, if when the
% type variables $\alphak$ are chosen to admit equality then $\tau$ also admits
% equality.
\label{tyfcn-lab}
A type function $\theta=\Lambda\alphak.\tau$\index{23.2}
 has the arity $\k$ as its kind; its bound variables must
be distinct.
A type function $\typefnc=\typefnctypefnc{\t[\K]}{\typefnc'}$\index{23.2}
 has kind $\Karr{\K}{\K'}$, provided $\typefnc'$ has kind $\K'$.

Two type functions are considered equal if they have the same kind
and differ in their choice of bound variables or
type names.
In particular, the equality attribute has no significance in a 
bound variable or type name of a type function;
for example, $\Lambda\alpha.\alpha\to\alpha$ 
(\typefnctypefnc{\t[\K]}{\apptypeapp{\t}{\typefnc}})
and $\Lambda\beta.\beta\to\beta$
 (\typefnctypefnc{\u[\K]}{\apptypeapp{\u}{\typefnc}}) are equal type functions
even if $\alpha$ ($\t$) admits equality but $\beta$($\u$) does not (provided
$\t,\u \not \in \TyNamesFcn \typefnc$).

If the type application \typeapp\ has kind \k\ then we
identify the type function $\typefnc = \typeapp$ with the type function
\typetypefnc{\alphak}{\apptype{\alphak}{\typeapp}}\ 
(provided $(\TyVarsFcn \alphak) \cap (\TyVarsFcn \typeapp) = \emptyset$) (eta-conversion).
If the type application \typeapp\ has kind \Karr{\K}{\K'} then we
identify the type function $\typefnc = \typeapp$ with the type function
\typefnctypefnc{\t[\K]}{\typeapptypefnc{\apptypeapp{\typeapp}{\typeapptypefnc{\t}}}}\
 (provided $\t\not \in \TyNamesFcn \typeapp$) (eta-conversion).

For convenience, when $\t$ has arity $k$, we 
shall write the type name $\t$ to mean
the type function $\typetypefnc{\alphak}{\apptype{\alphak}{\tynametypeapp{\t}}}$.

% cvr)

We write the application of a type function $\typefnc[\k]$ to a vector
$\tauk$ of types as $\tauk\theta$.
If $\typefnc=\typetypefnc{\alphak}{\tau}$ we set $\tauk\theta=\tau\{\tauk/\alphak\}$
(beta-conversion). 

We write $\tau\{\thetak/\t^{(\k)}\}$ for the result of substituting type
functions $\thetak$ for type names $\t^{(k)}$ in $\tau$.
We assume that all beta-conversions
are carried out after substitution, so that for example
\[(\tauk\t)\{\typetypefnc{\alphak}{\tau}/\t\}=\tau\{\tauk/\alphak\}.\]
(assuming $\t \not \in \TyNamesFcn \tauk$); and
\[(\typeapptypefnc{\apptypeapp{\t}{\typefnc}})\{\typefnctypefnc{\u}{\typefnc'}/\t\}=
     \typefnc'\{\typefnc/\u\}.\]
(assuming $\t \not \in \TyNamesFcn \typefnc$).
