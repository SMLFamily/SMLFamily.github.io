\section{Scope of Explicit Type Identifiers}
\label{scope-sec}

% In\index{23.10} the Core language, a type or datatype binding can 
% explicitly introduce type variables whose scope is that binding.
% In the modules, a description of a value, type, or datatype
% may contain explicit type variables whose scope is that
% description. However, we\index{23.11} still have to account for the
% scope of an explicit type variable occurring in the ``\ml{:} $\ty$'' 
% of a typed expression or pattern 
% or in the ``\ml{of} $\ty$'' of an exception binding. For the rest
% of this section, we consider such occurrences of type variables only.

% Every occurrence of a value declaration is said to
% {\sl scope} a set of explicit type variables determined as follows.

% %Every explicit type variable $\alpha$ is {\sl scoped at} a value binding
% %which is determined as follows.

% First, an occurrence of $\alpha$ in a value declaration $\valdec$ is said
% to be {\sl unguarded} if the occurrence is not part of a smaller value
% declaration within $\valbind$.
% In this case we say that $\alpha$ {\sl occurs unguarded} in the 
% value declaration.

% Then we say that $\alpha$ is {\sl scoped at} a particular occurrence
% $O$ of $\valdec$ in a program if 
% (1) $\alpha$ occurs unguarded in this value declaration, and 
% (2) $\alpha$ does not occur unguarded in any larger value declaration
% containing the occurrence $O$.\label{scope-def-lab}

% Hence, associated with every occurrence of a value declaration there is 
% a set $\U$ of the explicit type variables that are scoped at that
% occurrence. One may think of each occurrence of $\VAL$ as being implicitly
% decorated with such a  set, for instance:

% \vspace*{3mm}
% \halign{\indent$#$&$#$&$#$\cr
% \mbox{$\VAL_{\{\}}$ \ml{x = }}&\mbox{\ml{(}}&
% \mbox{\ml{let $\VAL_{\{\mbox{\ml{'a}}\}}$ Id1:'a->'a = fn z=>z in Id1 Id1 end,}}\cr
% & &\mbox{\ml{let $\VAL_{\{\mbox{\ml{'a}}\}}$ Id2:'a->'a = fn z=>z in Id2 Id2 end)}}\cr
% \noalign{\vspace*{3mm}}
% \mbox{$\VAL_{\{\mbox{\ml{'a}}\}}$ \ml{x = }}&\mbox{\ml{(}}&
% \mbox{\ml{let $\VAL_{\{\}}$ Id:'a->'a = fn z=>z in Id Id end,}}\cr
% & &\mbox{\ml{fn z=> z:'a)}}\cr}

% According to the inference rules in Section~\ref{stat-cor-inf-rules}
% the first example can be elaborated, but the second cannot since \ml{'a}
% is bound at the outer value declaration leaving no possibility of two 
% different instantiations of the type of \ml{Id} in the application
% \ml{Id Id}.

In\index{23.10} the Core language, a type or datatype binding can explicitly
introduce type identifiers whose scope is that binding.
Similary, in Modules, a type or datatype description can explicitly
introduce type identifiers whose scope is that description.
Moreover, in a
Core value declaration  \valdec,
 the sequence \tyidseq\ binds
type identifiers: a type identifier occurs free in \valdec\
iff it occurs free in \valbind\ and is not in the sequence
\tyidseq.
Similarly, in a Modules value specification \valspec,  
the sequence \tyidseq\ binds
type identifiers: a type identifier occurs free in \valspec\
iff it occurs free in \valdesc\ and is not in the sequence
\tyidseq.
However,\index{23.11} explicit binding of type identifiers at \ml{val}\ is
optional, so we still have to account for the scope of any
type indentifiers that occur free in type expressions.

Every occurrence of a value declaration or specification is said to {\sl scope} a set of explicit type identifiers determined as follows.

First, a free occurrence of $\alpha$ in a value declaration
\valdec\ or value specification \valspec\  is said to be {\sl unguarded} if the occurrence is not part
of a smaller value declaration or specification within the phrase.
In this case we say that {\sl $\alpha$ occurs unguarded} in the phrase.

Then we say that $\alpha$ is {\sl implicitly  scoped} at a particular value
declaration \valdec\ or value specification \valspec\
in a program if (1) $\alpha$ occurs
unguarded in this phrase, and (2) $\alpha$ does not occur
unguarded in any larger value declaration or specification containing the given phrase.
\label{scope-def-lab}

Henceforth, we assume that for every value declaration or specification
 $\VAL\ \tyidseq\cdots$
 occurring in the program, every explicit type identifier implicitly scoped at the 
$\VAL$ has been added to \tyidseq . Thus for example, in the two declarations

\begin{verbatim}
val x = let val id:'a->'a = fn z=>z in id id end 
val x = (let val id:'a->'a = fn z=>z in id id end; fn z=>z:'a)
\end{verbatim}
the type identifier \verb+'a+ is scoped differently; they become respectively

\begin{verbatim}
val x = let val 'a id:'a->'a = fn z=>z in id id end
val 'a x = (let val id:'a->'a = fn z=>z in id id end; fn z=>z:'a)
\end{verbatim}

Then, according to the inference rules in Section~\ref{stat-cor-inf-rules} the first example can be elaborated, but the second cannot since
\verb+'a+ is bound at the outer value declaration leaving no possibility of two different instantiations of the type of
\verb+id+ in the application \verb+id id+.

