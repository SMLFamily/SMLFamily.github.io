\section{Appendix: The Development of ML}
\label{story-app}

This Appendix records the main stages in the development
of ML, and the people principally involved.  The main emphasis is upon
the design of the language; there is also a section devoted to 
implementation. On the other hand, no attempt is made to record work on 
implementation environments, or on applications of the language.
 
\subsection*{Origins}

ML and its semantic description have evolved over a period of about
fourteen years.  It is a fusion of many ideas from many people;  in this
appendix we try to record and to acknowledge the important precursors
of its ideas, the important influences upon it, and the important
contributions to its design, implementation and semantic description.

ML, which stands for {\sl meta language}, was conceived as a medium for
finding and performing proofs in a formal logical system.  This
application was the focus of the initial design effort, by Robin Milner
in collaboration first with Malcolm Newey and Lockwood Morris, then
with Michael Gordon and Christopher Wadsworth \cite{GMMNW}. The intended application
to proof affected the design considerably.  Higher order functions 
in full generality seemed necessary for programming proof tactics and
strategies, and also a robust type system (see below).  At the same time,
imperative features were important for practical reasons; no-one had experience
of large useful programs written in a pure functional style. In particular,
an exception-raising mechanism was highly desirable for the natural presentation
of tactics.

The full definition of this first version of ML was included in a book 
\cite{GMW}
which describes LCF, the proof system which ML was designed to support. 
The details of how the proof application exerted an influence on design is
reported by Milner \cite{Mil2}.
Other early influences were the applicative languages already in use
in Artificial Intelligence, principally LISP \cite{McC}, ISWIM \cite{Lan} and 
POP2 \cite{BP}.

\subsection*{Polymorphic types}

The polymorphic type discipline and the associated type-assignment algorithm
were prompted by the need for security;  it is vital to know that when
a program produces an object which it claims to be a theorem, then it
is indeed a theorem.  A type discipline provides the security, but a
polymorphic discipline also permits considerable flexibility.

The key ideas of the type discipline were evolved in
combinatory logic by Haskell Curry and Roger Hindley, who arrived at
different but equivalent algorithms for computing principal type schemes.
Curry's \cite{Cur} algorithm was by equation-solving; Hindley \cite{Hin} used
the unification algorithm of Alan Robinson \cite{Rob} and  also presented the 
precursor of our type inference system.
James Morris \cite{Mor} independently gave an equation-solving algorithm very
similar to Curry's.  The idea of an algorithm for finding principal 
type schemes is very natural and may well have been known earlier. I am
grateful to Roger Hindley for pointing out that Carew Meredith's inference 
rule for propositional logic called Condensed Detachment, defined
in the early 1950s, clearly suggests that he knew such an algorithm \cite{Mer}. 

Milner \cite{Mil1}, during the design of ML,
rediscovered principal types and their calculation by unification, for a
language (slightly richer than combinatory logic) containing local
declarations.  He and Damas \cite{DM} presented the ML type inference systems
following Hindley's style.
Damas \cite{Dam}, using ideas from Michael Gordon, also devised the first
mathematical treatment of polymorphism in the presence of references and
assignment.  Tofte \cite{Tof-a} produced a different scheme, which
has been adopted in the language.

\subsection*{Refinement of the Core Language}

Two movements led to the re-design of ML.  One was the work of Rod
Burstall and his group on specifications, crystallised in the specification
language CLEAR \cite{BG} and in the functional programming language HOPE
\cite{BMS};  the latter was for expressing executable specifications.  The
outcome of this work which is relevant here was twofold.  First, there were
elegant programming features in HOPE, particularly pattern matching
and clausal function definitions;  second, there were ideas on modular
construction of specifications, using signatures in the 
interfaces.  A smaller but significant movement was by Luca Cardelli,
who extended the data-type repertoire in ML by adding named records and
variant types.

In 1983, Milner (prompted by Bernard Sufrin) wrote the first draft of a
standard form of ML attempting to unite these ideas;  over the next three
years it evolved into the Standard ML Core Language.  Notable here was
the harmony found among polymorphism, HOPE patterns and Cardelli records,
and the nice generalisations of ML exceptions due to ideas from Alan
Mycroft, Brian Monahan and Don Sannella.  A simple stream-based I/O mechanism
was developed from ideas of Cardelli by Milner and Harper.  The Standard ML
Core Language is described in detail in a composite report \cite{HMM} which also
contains a description of the I/O mechanism and MacQueen's proposal
for program modules (see later for discussion of this). Since then only
few changes to the Core Language have occurred.  Milner proposed
equality types, and these were added, together with a few minor
adjustments \cite{Mil3}.   The latest and final development has been in the
exception mechanism, by MacQueen using an idea from Burstall \cite{AMMT};  
it unites the ideas of exception and data type construction.

\subsection*{Modules}

Besides contributory ideas to the Core Language, HOPE \cite{BMS} contained
a simple notion of program module.  The most important and original
feature of ML Modules, however, stems from the work on parameterised
specifications in CLEAR \cite{BG}.  MacQueen, who was a member of Burstall's
group at the time, designed \cite{Mac} a new parametric module feature for HOPE
inspired by the CLEAR work.
He later extended the parameterisation ideas by a novel method
of specifying sharing of components among the structure parameters of
a functor, and produced a draft design which accommodated features already
present in ML -- in particular the polymorphic type system.  This design
was discussed in detail at Edinburgh, leading to MacQueen's first
report on Modules \cite{HMM}.

Thereafter, the design came under close scrutiny through a draft operational
static semantics and prototype
implementation of it by Harper, through Kevin Mitchell's implementation of
the evaluation, through a denotational semantics written by
Don Sannella, and then through further work on operational semantics by Milner
and Tofte.  (More is said about this in the later section on Semantics.)
In all of this work the central ideas withstood scrutiny, while it
also became clear that there were gaps in the design and ambiguities in
interpretation.  (An example of a gap was the inability to specify sharing
between a functor argument structure and its result structure;  an
example of an ambiguity was the question of whether sharing exists in a
structure over and above what is specified in the signature expression
which accompanies its declaration.)

Much discussion ensued;  it was possible for a wider group to comment on
Modules through using Harper's prototype implementation, while
Harper, Milner and Tofte gained understanding during development of this
semantics.  In parallel, Sannella and Tarlecki explored the implications
of Modules for the methodology of program development \cite{ST}.  Tofte, in his 
thesis \cite{Tof}, proved several technical
properties of Modules in a skeletal language, which generated considerable
confidence in this design. A key point in this development was the proof
of the existence of principal signatures, and, in the careful distinction
between the notion of {\it enrichment} of structures, which allows more
polymorphism and more components, and {\it realisation} which allows more
sharing.

At a meeting in Edinburgh in 1987 a choice of two designs was presented,
hinging upon whether or not a functor application should coerce its
actual argument to its argument signature.  The meeting chose coercion,
and thereafter the production of Section~\ref{statmod-sec} of this report -- the
Static Semantics of Modules -- was a matter of detailed care.  That
section is undoubtedly the most original and demanding part of this
semantics, just as the ideas of MacQueen upon which it is based are the
most far-reaching extension to the original design of ML.

\subsection*{Implementation}

The first implementation of ML was by Malcolm Newey, Lockwood Morris
and Robin Milner in 1974, for the DEC10.  Later Mike Gordon and Chris
Wadsworth joined;  their work was mainly in specialising ML towards
machine-assisted reasoning.  Around 1980 Luca Cardelli implemented
a version on VAX;  his work was later extended by Alan Mycroft, Kevin
Mitchell and John Scott.  This version contained one or two new data-type 
features, and was based upon the Functional Abstract Machine (FAM), a virtual
machine which has been a considerable stimulus to later implementation.
By providing a reasonably efficient implementation, this work enabled the 
language to be taught to students; this, in turn, prompted
the idea that it could become a useful general purpose language.

In Gothenburg, an implementation was developed by Lennart Augustsson and 
Thomas Johnsson in 1982, using lazy 
evaluation rather than call-by-value; the result was called Lazy ML
and is reported in \cite{Aug}.  This work is part of continuing research
in many places on implementation of lazy evaluation in pure functional 
languages.  But for ML, which includes exceptions and assignment, the 
emphasis has been mainly upon strict evaluation (call-by-value).

In Cambridge, in the early 1980s, Larry Paulson made considerable improvements
to the Edinburgh ML compiler, as part of his wider programme of improving
Edinburgh LCF to become Cambridge LCF \cite{Pau}.  This system has supported 
larger proofs than the Edinburgh system, and with greater convenience; in 
particular, the compiled ML code ran four to five times faster.

Around the same time G\'{e}rard Huet at INRIA (Versailles) adapted ML to 
Maclisp on Multics, again for use in machine-assisted proof.  There was
close collaboration between INRIA and Cambridge in this period.
ML has undergone a separate development in the group at INRIA, arriving at a 
language and implementation known as CAML \cite{CCM}; this is close to the core 
language of Standard ML, but does not include the Modules. 

The first implementation of the Standard ML core language was by Mitchell, 
Mycroft and John Scott of Edinburgh, around 1984, and this was shortly followed
by an implementation by David Matthews at Cambridge, carried out 
in his language Poly.  

The prototype implementation of Modules, before that part
of the language settled down, was done in 1985-6;  Mitchell dealt with
evaluation, while Harper tackled the elaboration (or `signature checking')
which raised problems of a kind not previously encountered.  The Edinburgh
implementation continues to play the role of a test-bed for language
development.  

Meanwhile Matthews' Cambridge implementation also advanced to embrace
Modules. This implementation has 
supported applications of considerable size, both for machine-assisted proof
and for hardware design. 

In 1986, as the Modules definition was settling down, David MacQueen began 
an implementation at Bell AT\&T Laboratories, joined later by Andrew Appel
and Trevor Jim who are particularly interested in compilation into high quality
machine code.

The Bell and Cambridge implementations, the former led by MacQueen and Appel,
the latter by Matthews, are currently the most complete and highly engineered.
Other currently active implementations are by Michael Hedlund at the
Rutherford-Appleton Laboratory, by Robert Duncan, Simon Nichols and
Aaron Sloman at the University of Sussex (POPLOG) and by Malcolm Newey
and his group at the Australian National University.
 
\subsection*{Semantics}

The description of the first version of ML \cite{GMW} was informal, and in an 
operational style;  around the same time a denotational semantics was written,
but never published, by Mike Gordon and Robin Milner.  Meanwhile structured
operational semantics, presented as an inference system, was gaining
credence as a tractable medium.  This originates with the reduction rules
of $\lambda$-calculus, but was developed more widely through the work
of Plotkin \cite{Plo}, and also by Milner.   This was at
first only used for dynamic semantics, but later the
benefit of using inference systems for both static and dynamic
semantics became apparent.  This advantage was realised when Gilles Kahn
and his group at INRIA were able to execute early versions of both forms of 
semantics for the ML Core Language using their Typol system \cite{Des}.  
The static and dynamic
semantics of the Core reached a final form mostly through work by
Mads Tofte and Robin Milner.

The modules of ML presented little difficulty as far as dynamic semantics
is concerned, but the static semantics of Modules was a concerted effort
by several people.  MacQueen's original informal description \cite{HMM}
was the starting point;  Sannella wrote a denotational semantics for
several versions, which showed that several issues had not been settled
by the informal description.  Robert Harper, while writing the first
implementation of Modules, made the first draft of the static semantics.
Harper's version made clear the importance of structure names; 
work by Milner and Tofte introduced further ideas including realisation;
thereafter a concerted effort by all three led to several suggestions
for modification of the language, and a small range of alternative
interpretations;  these were assessed in discussion with MacQueen, and
more widely with the principal users of the language, and an agreed form
was reached.


There is no doubt that the interaction between design and semantic
description of  Modules has been one of the most striking phases in the
entire language development, leading (in the opinion of those involved)
to a high degree of confidence both in the language and in the semantic
method.

\subsection*{Literature} 
The present document is the definition of Standard ML; further versions
of it will be produced as the language develops (but the intention is
to minimise the number of versions).  An informal definition, consistent
with Version~2 of this document as far as the Core Language is concerned,
is provided by \cite{HMM}, as modified by \cite{Mil3} and \cite{AMMT}. An elementary
textbook covering the Core language has been recently published, 
written by {\AA}ke Wikstr\"{o}m \cite{Wik}.  
Robert Harper \cite{Har} has written a shorter
introduction which also includes material on Modules.

\subsection*{Further acknowledgments}
Apart from the people mentioned above we also acknowledge the following,
all of whom have contributed in some way to the evolution of ML:
Guy Cousineau, Simon Finn, Jim Hook,
Gerard Huet, Gilles Kahn,
Brian Monahan, 
Peter Mosses, Alan Mycroft,  David Park,
David Rydeheard, 
David Schmidt,  Stefan Sokolowski, Bernard Sufrin,
Philip Wadler.

\newpage
%\subsection*{References}
\addcontentsline{toc}{section}{\protect\numberline{}{\vrule width0pt height2cm depth0pt References}}
\label{references-sec}
\begin{thebibliography}{99}

\bibitem{AMMT} Appel, A., MacQueen, D.B., Milner, R. and Tofte, M., 
{\em Unifying Exceptions with Constructors in Standard ML}, Report ECS-LFCS-88-55,
Laboratory for Foundations of Computer Science, 
Computer Science Dept, Edinburgh University, 1988.

\bibitem{Aug} Augustsson L. and Johnsson, T., {\em Lazy ML User's Manual}, 
Dept.\ of Computer Sciences, Chalmers University of Technology,
Gothenburg, 1987.

\bibitem{BMS} Burstall R.M., MacQueen, D.B. and Sannella, D.T., 
{\em HOPE: An Experimental
Applicative Language}, Report CSR-62-80, 
Computer Science Dept, Edinburgh University, 1980.

\bibitem{BG} Burstall, R.M. and Goguen, J.A., {\em Putting Theories together to
make Specifications}, Proc Fifth Annual Joint Conference on Artificial
Intelligence, Cambridge, Mass., 1977, pp 1045--1058.
\bibitem{BP} Burstall, R.M. and Popplestone, R., {\em POP-2 Reference Manual},
Machine Intelligence 2, ed Dale and Michie, Oliver and Boyd, 1968.

\bibitem{CCM} Cousineau, G., Curien, P.L. and Mauny, M., {\em The Categorical
Abstract Machine}, in Functional Programming Languages and Computer 
Architecture, ed Jouannaud, Lecture Notes in Computer Science Vol 201,
Springer Verlag, 1985, pp 50--64.

\bibitem{Cur} Curry, H.B., {\em Modified Basic Functionality in Combinatory
Logic}, Dialectica 23, 1969, pp 83--92.

\bibitem{Dam} Damas, L., {\em Type Assignment in Programming Languages}, PhD
thesis, CST-33-85, Computer Science Department, Edinburgh University, 1985.

\bibitem{DM} Damas, L. and Milner, R., {\em Principal Type-schemes for
Functional Programs}, Proc 9th annual symposium on Principles of Programming
Languages, ACM, 1982.

\bibitem{Des} Despeyroux, T., {\em Executable Specification of Static Semantics},
Proc Symposium on Semantics of Data Types, Sophia Antipolis,
Springer-Verlag Lecture Notes in Computer Science, Vol.173, 1984.

\bibitem{GMMNW} Gordon, M.J.C., Milner, R., Morris,L., Newey, M.C. and 
Wadsworth, C.P., {\em A Metalanguage for Interactive Proof in LCF}, 
Proc 5th ACM Symposium on Principles of
Programming Languages, Tucson, 1978.
 
\bibitem{GMW} Gordon, M.J.C., Milner, R. and Wadsworth, C.P., 
{\em Edinburgh LCF: a Machanised Logic of Computation},
Springer-Verlag Lecture Notes in Computer Science, Vol.78, 1979.

\bibitem{Har} Harper, R.W., {\em Introduction to Standard ML}, Report
ECS-LFCS-86-14, 
Laboratory for Foundations of Computer Science, 
Computer Science Department, Edinburgh University, 1986.

\bibitem{Hin} Hindley, R., {\em The Principal Type-scheme of an Object in
Combinatory Logic}, Transactions of AMS 146, pp29--60, 1969.

\bibitem{HMM} Harper, R.M., MacQueen, D.B. and Milner, R., 
{\em Standard ML},
Report ECS-LFCS-86-2, 
Laboratory for Foundations of Computer Science, 
Computer Science Department, Edinburgh University, 1986.

\bibitem{ML1} Harper, R.M., Milner, R., Tofte, M., 
{\em The Semantics of Standard ML, Version 1}
Report ECS-LFCS-87-36,
Laboratory for Foundations of Computer Science, 
Computer Science Department, Edinburgh University, 1987.
 
\bibitem{ML2} Harper, R.M., Milner, R., Tofte, M., 
{\em The Definition of Standard ML, Version 2}
Report ECS-LFCS-88-62,
Laboratory for Foundations of Computer Science, 
Computer Science Department, Edinburgh University, 1988.
 
\bibitem{ML3} Harper, R.M., Milner, R., Tofte, M., 
{\em The Definition of Standard ML, Version 3}
Report ECS-LFCS-89-81,
Laboratory for Foundations of Computer Science, 
Computer Science Department, Edinburgh University, 1989.
 
\bibitem{Lan} Landin, P.J., {\em The next 700 Programming Languages}, CACM, Vol.9,
 No.3, 1966, pp57--164.

\bibitem{Mac} MacQueen, D.D., {\em Structures and parameterisation in a typed 
functional language}, Proc.\ Symposium on Functional Programming and Computer
Architecture, Aspinas, Sweden, 1981.

\bibitem{McC} McCarthy, J. et al., {\em LISP 1.5 Programming Manual}, 
The MIT Press, Cambridge, Mass, 1956.

\bibitem{Mer} Meredith, D., {\em In memoriam Carew Arthur Meredith}, Notre Dame
J. Formal Logic, Vol 18, 1977, pp 513--516.

\bibitem{Mil1} Milner, R.,
 {\em A theory of type polymorphism in programming},
J. Comp. Sys.Sci, Vol 17, 1978, pp 348--375.

\bibitem{Mil2} Milner, R., {\em How ML Evolved}, 
Polymorphism (The ML/LCF/Hope Newsletter),
Vol.1, No.1, 1983.

\bibitem{Mil3} Milner, R., {\em Changes to the Standard ML Core Language},
Report ECS-LFCS-87-33,
Laboratory for Foundations of Computer Science, 
Computer Science Department, Edinburgh University, 1987.

\bibitem{Mor} Morris, J.H., {\em Lambda Calculus Models of Programming Languages},
MAC-TR-57 (Thesis), Project MAC, M.I.T., 1968.

\bibitem{Pau} Paulson, L.C., {\em Logic and Computation: Interactive Proof with
LCF}, Cambridge Tracts in Theoretical Computer Science 2, Cambridge University
Press, 1987.

\bibitem{Plo} Plotkin, G.D., {\em A Structural Approach to Operational Semantics},
Technical Report DAIMI FN-19, Computer Science Department, \AA rhus University,
1981.

\bibitem{Rob} Robinson, J.A., {\em A Machine-oriented Logic based upon the
Resolution Principle}, Journal of ACM, Vol 12, No 1, pp23-41, 1965. 

\bibitem{ST} Sannella, D.T. and Tarlecki, A., {\em Program Specification
and Development in Standard ML}, 
Proc 12th ACM Symposium on Principles of
Programming Languages, New Orleans, 1985.

\bibitem{Tof} Tofte, M., {\em Operational Semantics and Polymorphic Type
Inference}, PhD Thesis CST-52-88, 
Computer Science Department, Edinburgh University, 1988. (Also appears as
Report ECS-LFCS-88-54 of the Laboratory for Foundations of Computer Science.)

\bibitem{Tof-a} Tofte, M., {\em Type Inference for Polymorphic References}
(To appear in Information and Computation)

\bibitem{Wik} Wikstr\"{o}m, \AA., {\em Functional Programming using Standard
ML}, Prentice Hall, 1987.
\end{thebibliography}




